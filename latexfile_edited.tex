\documentclass[a4paper, titlepage]{article}
\usepackage[utf8]{inputenc}

\usepackage{graphicx}
\usepackage{amsfonts}
\usepackage{algorithm,algpseudocode}
\usepackage{listings} % för uppgift 3
\renewcommand{\thesubsubsection}{\thesection.\alph{subsubsection}}
\usepackage{minted}
\usepackage[version=4]{mhchem} %  for H2O
\usepackage{xcolor}
\usepackage{tikz}

\chapter{plain}
Hej majid detta är första meningen. Nu kommer andra meningen.


\title{Laboration 2\\Grunder i \LaTeX, MA1475}% majids kommentar
\author{Majid Mohamed Hamid}
\date{VT 2021}
\begin{document}

\maketitle
\begin{center} 
Laborationsuppgifter 
\end{center}

\section{Hantera referenser}
\begin{thebibliography}{0}
\bibitem{algoritms} Jeffrey H. Kingston, \emph{Algoritms and Data Structures}, second edition,\\ Addison-Wesley, Harlow, England, 1997.

\bibitem{cyber security} Tomasz Surmacs and Anders Carlsson, \emph{Cyber Security for Next Generation\\Experts}, Exakta print AB, Malmö, 2018.

\bibitem{Databasteknik} Tomas Padron-McCarthy and Tore Risch and studentlitteratur, \\\emph{Databasteknik}, studentlitteratur AB, Lund, 2014. 
\end{thebibliography}

\section{Felhantering}
\subsection*{Introduktion}
Låt $k$ vara ett heltal och sätt $n = 2k$. Då är
\begin{equation}\label{eq:ekv}
(-1)^n = (-1)^{2k} = ((-1)^2)^k = 1^k = 1.
\end{equation}
Från \ref{eq:ekv} och $i^2 = -1$ följer att $i^{4k} = (i^2)^{2k} = 1.$ 
Vidare följer från $\sqrt{2} \notin \mathbb{Q}$ att


\section{Typsättning av programkod}
\begin{minted}{python}
def f(a, b):
  if a == b:
      return a
  elif a %  2 == 0:
      if b %  2 == 0:  #både a och är jämna
          return 2 * f(a//2, b//2)
      else:           #a är jämn och b är udda
          return f(a//2, b)
  else:
      if b %  2 == 0:  #a är udda och b är jämn
          return f(a, b//2)
      else:           #både a och b är udda
          if a > b:
              return f(a - b, b)
          else:       #a < b
              return f(a, b - a)
\end{minted}                

\newpage
\section{Programmering}
\subsubsection{}
\newcommand{\vatten}[1]{% 
Den kemiska formeln #1 beskriver den kemiska föreningen vatten.% 
}
\vatten{\ce{H2O}}

\subsubsection{}% b
\begin{figure}
\centering
\includegraphics[scale=0.1]{majid.jpg}% en bild som jag kan referera till
\caption{Nyårsafton 2021}
\label{minfigur}
\end{figure}

\newcommand{\figref}[1]{% 
se figur \ref{#1} på sidan \pageref{#1}% 
}
\figref{minfigur}

\subsubsection{}% c
\newcommand{\VIKTIGT}[1]{% 
\fcolorbox{red}{red}{\textcolor{yellow}{\MakeUppercase{\textbf{\underline{#1}}}}}% 
}
\VIKTIGT{Bismillah alrahman alrahim}

\subsubsection{}% d
\newcommand{\JB}[2]{% 
My name is #2, #1 #2.% 
}
\JB{James}{Bond}

\subsubsection{}% e
\newcommand{\coolpil}[4]{
\begin{tikzpicture}
\draw[color = #2, fill = #1, very thick][scale=0.5] (0,0)--(5,0)--(5,-1)--(7,1)--(5,3)--(5,2)--(0,2)--(0,0);
\node[right] at (0,0.5){\textcolor{#3}{#4}};
\end{tikzpicture}
}
\coolpil{yellow}{red}{blue}{\LARGE Utgång}
\coolpil{cyan}{blue}{black}{\Large VÄNSTER}
\end{document}